\documentclass{article}
\usepackage[utf8]{inputenc}
\usepackage{caption}
\usepackage{xcolor}

\title{
    \textbf{ \begin{center} \Huge Cahier des charges \end{center} }
    \textrm{ \underline{Groupe:} Sépanou }
}

\author{
    LO Jean-Pierre | MICHALON Loïc \and
    MICHOT Maxence | TCHEKACHEV David
}

\date{Janvier 2021}

\definecolor{dark_grey}{RGB}{24, 26, 27}
\pagecolor{dark_grey}
\color{white}

\begin{document}

\maketitle

\pagebreak

\renewcommand*\contentsname{\textbf{\Huge Sommaire \newline}}
\large \tableofcontents

\pagebreak
\normalsize
\section{Introduction}

...

\pagebreak

\section{Le groupe et ses membres}

\subsection{Le groupe}

...

\subsection{Les membres}

\subsubsection{LO Jean-Pierre}

J'ai commencé à m'intéresser à l'informatique au début de mon année de 4ème marquée par l'achat de mon premier ordinateur.\\
Auparavant, j'ai eu l'occasion de jouer à divers jeux sur DS puis sur PC (Minecraft bien sûr) qui, finalement, ont éveillé en moi cet intérêt pour l'informatique et plus précisément pour la programmation.
Ainsi ai-je pu m'essayer à divers langages tels que Java, Python ou le C, qui m'ont permis de découvrir la plupart des paradigmes de programmation. J'ai également eu l'occasion de m'intéresser au web dans le cadre de petits projets d'ICN ou d'ISN. \\
Néanmoins, je n'ai jamais réellement approfondi un langage en particulier... \\
D'autre part, j'apprécie tout particulièrement me plonger dans le jeu vidéo à mes heures perdues avec une préférence certaine pour ceux alliant exploration, survie, monde ouvert et histoires captivantes. Les jeux rogue-like n'étant pas dans mon répertoire mais reprenant toutefois des mécaniques similaires, le choix de ce type de jeu comme projet de S2 me semblait attirant. \\
Finalement, je suis content que nous nous soyons orientés vers la création d'un jeu vidéo car cela m'apportera des connaissances nouvelles accompagnées de l'expérience singulière du travail en groupe.

\subsubsection{MICHALON Loïc}

J'ai toujours aimé l'informatique. En effet, cela m'intéresse depuis que je suis tout petit notamment depuis la primaire. Cependant j'ai commencé à toucher au code en Seconde avec le Javascript mais n'ayant aucun projet à réaliser j'ai vite arrêté mais cela resta très formateur. J'ai donc surtout codé sur CASIO durant la Seconde et la Première, le temps me manquant en Terminale. Ainsi rentrer à l'EPITA m'a permis de me mettre pour de bon à l'informatique pour mon plus grand bonheur. \\
J'aime jouer aux jeux vidéos durant mon temps libre surtout aux jeux de stratégie ou aux jeux avec une grande rejouabilité tel que les rogue-like. En dehors des jeux vidéos j'aime bien m'instruire sur le monde extérieur en Histoire comme en géopolitique et j'affectionne tout particulièrement les sciences "dures". \\
Ce projet va me permettre d'avoir au moins fait un jeu dans ma vie. De plus, ce jeu fait parti d'un genre (rogue-like) que j'apprécie fortement grâce au jeu vidéo \textit{Enter the Gungeon} et ce projet va donc me permettre de découvrir les coulisses de la programmation en équipe ce qui me semble très intéressant et formateur. Je regarde donc avec intérêt la suite de ce projet en espérant pouvoir le mener à bien lors du S2.

\subsubsection{MICHOT Maxence}

Les jeux vidéo ont toujours été un refuge et un moyen d'expression pour moi, de plus j'ai été mis en contacte avec l'informatique dès le plus jeune âge. C'est dans ces conditions que je me suis intéresse à la programmation vers la fin du collège notamment. D'abord avec scratch puis directement avec des langages bas niveaux comme le TI Basic et surtout le C/C++. \\
Dès lors passionné par la programmation, Je m'intéresse à toutes sortes de langages différents, qu'ils soit orienté objet ou même ésotériques. Bien qu'ayant fait un peu de développement web, je me spécialise dans la programmation bas niveau ainsi que tout récemment dans le développement de jeux comme par exemple pour mon projet d'informatique de Terminal. \\
Ce projet est pour moi l'opportunité de me plonger plus en profondeur dans la programmation de jeux et de découvrir des outils comme Unity. Il est de plus une occasion d'étoffer mon expérience des projets de groupe.

\subsubsection{TCHEKACHEV David}

J'ai eu la chance d'être plongé dans l'informatique dès le plus jeune âge, tout d'abord avec un ordinateur familial, puis grâce à une association qui enseigne l'informatique aux enfants à Paris.\\
Depuis, je programme tout ce qui me passe par l'esprit, souvent pour moi ou mes amis. Je suis resté longtemps dans l'univers Minecraft à programmer des modules pour enrichir le jeu (plugins). Mais après plusieurs années, je m'en suis lassé et j'ai quitté cet univers. \\
Je me suis ensuite lancé dans la création de bots pour automatiser certaines tâches de la vie, ainsi que des sites web pour gérer certains systèmes de façon informatique. Mes derniers sites servent à gérer des tournois sportifs ainsi qu'un site qui gère les voeux d'orientation de lycéens.
Pour le projet de S2, je suis curieux de découvrir le monde du jeu vidéo au plus bas niveau, là où il est créé et imaginé. \\
Je pense que la diversité de notre groupe nous permet d'avoir un équilibre afin de puiser dans les compétences de chacun afin de réaliser un projet bien réparti.

\pagebreak

\section{Le projet}

\subsection{Origine et nature du projet}

D’où vient l’idée de ce projet, de quelle nature est il ? Jeu, Utilitaire, Traitement d’images, etc...

\subsubsection{Idée initiale}

...

\subsubsection{Nature du jeu}

...

\subsection{Objet de l'étude}

Quels sont les buts et intérêts de ce projet. Qu’est-ce que cela peut vous apporter en groupe ou individuellement ?

\subsection{État de l'art}

L'un des premiers jeux de ce type est le jeu \textit{Rogue} sorti en 1980 qui est un sous-genre du Jeu de Rôle (JDR ou RPG en anglais) qui fut le pionnier de ce sous-genre. Il consiste en un jeu où il faut explorer un donjon généré de façon procédurale en tuant des ennemis pour gagner de l'expérience et de l'équipement, toute mort étant définitive. Ainsi, les jeux rogue-like sont surtout caractérisés par des niveaux fermés, générés procéduralement à chaque nouvelle partie, et dont le but est d'atteindre la fin sans mourir ; chaque mort nous faisant recommencer une partie du début. Cependant, aujourd'hui une très grande majorité des rogue-like a introduit un système de progression entre chaque partie comme des nouveaux objets à débloquer, des raccourcis vers les niveaux suivants, ou encore des boss entre chaque niveau.\\
\\
Ainsi, le genre du rogue-like existe encore aujourd'hui et n'est pas en perte de vitesse. Un des plus récents et célèbres rogue-like est \textit{The Binding of Isaac} ainsi que ses suites ou extensions qui correspondent au genre du rogue-like ; le but étant d'aller au bout des niveaux générés procéduralement en ramassant des équipements différents à chaque partie.\\
Un autre rogue-like connu est le jeu vidéo \textit{Enter the Gungeon} qui correspond au genre du rogue-like dont le but est de vaincre le \textit{Dragun}, gardien du \textit{Gungeon}. Contrairement aux autres jeux rogue-like, ce jeu est également un bullet-hell (genre caractérisant des jeux où il y a énormément de projectiles ennemis à éviter sur l'écran) car il repose sur l'univers des armes à feu et, contrairement aux autres rogue-like, le joueur et les ennemis attaquent à distance.\\
Enfin, le genre du rogue-like est toujours d'actualité avec la sortie en Septembre 2020 du jeu \textit{Hades} salué par la critique et nominé aux \textit{Steam Awards} de 2020. Dans ce jeu, on incarne le fils d'Hadès qu'on peut améliorer de partie en partie grâce aux ressources récoltées lors de chacune d'elles. On peut également jouer des armes différentes selon nos goûts et l'on se fait aider par les dieux de l'Olympe afin de réussir à terminer le jeu.\\
\\
Ainsi, le genre du rogue-like est caractérisé surtout par sa rejouabilité quasi-infinie due à sa génération procédurale et son genre très enclin à intégrer de très nombreux secrets dans le jeu pour atteindre le 100\%. Les grands aspects du rogue étant donc :\\
\begin{itemize}
    \item Une génération procédurale des niveaux
    \item De nombreux ennemis variés
    \item Un aspect RPG
    \item De nombreux secrets
    \item Une grande durée de vie
\end{itemize}

%Quel est le premier logiciel/jeu de ce type ? Quels sont les principaux autres logiciels/jeux de ce type existants (vous en citerez au minimum trois) ? Quels sont leurs points forts ? Quelles sont leurs fonctionnalités propres ?

\subsection{Découpage du projet}

Qui, dans l’équipe qui réalisera ce projet fera quoi ? Partage des tâches,mais aussi découpage du projet en différentes parties si cela s’avère nécessaire. Par exemple pour un logiciel type Blender, il y aurait l’éditeur 2D, l’éditeur 3D, le gestionnaire de matériel,l’animateur, etc... \\
$\rightarrow$ $1^{er}$ Rogue Like $\rightarrow$ Rogue, 1980 \\
$\rightarrow$ Isaac, FTL, ETG, Hadès

\pagebreak

\subsubsection{Répartition des tâches}

\begin{table}[h!]
    \centering
    \caption*{Répartition des tâches par personne}
    \begin{tabular}{ |c|c|c|c|c| }
        \hline
        Tâches & Jean-Pierre & Loïc & Maxence & David \\
        \hline
        Progammation du jeu & - & - & - & - \\
        \hline
        Intelligence artificielle & - & - & - & - \\
        \hline
        Architecture des niveaux & - & - & - & - \\
        \hline
        Graphismes \& musiques & - & - & - & - \\
        \hline
        Réseau & - & - & - & - \\
        \hline
        Site Internet & - & - & $\times$ & $\otimes$ \\
        \hline
    \end{tabular}
    \caption*{
        \\ $\otimes$ : Responsable
        \\ $\times$ : Suppléant
    }
    \label{table:repartition}
\end{table}

\subsubsection{Logiciels et matériel utilisé}

...

\subsubsection{Le réseau}

Le jeu étant voué à être multijoueur, même si un mode solo sera disponible, il y aura deux entités séparées:

\begin{itemize}
    \item Le client:  Il sera chargé d'afficher le monde et interpréter les actions du joueur sur l'ordinateur. Il sert donc d'interface entre le serveur et le joueur.
    
    \item Le serveur: Il sera chargé de générer le monde, sauvegarder les parties des joueurs, interconnecter les joueurs lors des sessions multijoueurs et c'est lui qui va gérer la logique et le fonctionnement du jeu.
\end{itemize}

Lors des sessions solo, un serveur local est lancé sur l'ordinateur du joueur afin de garder la même base de code entre le mode solo et le mode multijoueur. \\
Ces deux entités communiquent entre elles en réseau, et s'échangent des données en continu pendant toute la partie.

\subsubsection{L'intelligence artificielle}

...

\end{document}
