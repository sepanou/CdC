\documentclass{article}
\usepackage[utf8]{inputenc}
\usepackage{caption}
%\usepackage{xcolor}

\title{
    \textbf{ \begin{center} \Huge Cahier des charges \end{center} }
    \textrm{ \underline{Groupe:} Sépanou }
}

\author{
    LO Jean-Pierre | MICHALON Loïc \and
    MICHOT Maxence | TCHEKACHEV David
}

\date{Janvier 2021}

%\definecolor{dark_grey}{RGB}{24, 26, 27}
%\pagecolor{dark_grey}
%\color{white}

\begin{document}

\maketitle

\pagebreak

\renewcommand*\contentsname{\textbf{\Huge Sommaire \newline}}
\large \tableofcontents

\pagebreak
\normalsize
\section{Introduction}

...

\pagebreak

\section{Le groupe et ses membres}

\subsection{Le groupe}

...

\subsection{Les membres}

\subsubsection{LO Jean-Pierre}

J'ai commencé à m'intéresser à l'informatique au début de mon année de 4ème marquée par l'achat de mon premier ordinateur. \\
Auparavant, j'ai eu l'occasion de jouer à divers jeux sur DS puis sur PC (Minecraft bien sûr) qui, finalement, ont éveillé en moi cet intérêt pour l'informatique et plus précisément pour la programmation.
Ainsi ai-je pu m'essayer à divers langages tels que Java, Python ou le C, qui m'ont aidés à découvrir la plupart des paradigmes de programmation. J'ai également eu l'occasion de m'intéresser au web dans le cadre de petits projets d'ICN ou d'ISN. \\
Néanmoins, je n'ai jamais réellement approfondi un langage en particulier... \\
D'autre part, j'apprécie tout particulièrement me plonger dans le jeu vidéo à mes heures perdues avec une préférence certaine pour ceux alliant exploration, survie, monde ouvert et histoires captivantes. Les jeux rogue-like n'étant pas dans mon répertoire mais reprenant toutefois des mécaniques similaires, le choix de ce type de jeu comme projet de S2 me semblait attirant. \\
Finalement, je suis content que l'on se soit orientés vers la création d'un jeu vidéo car cela m'apportera des connaissances nouvelles accompagnées de l'expérience singulière qu'apporte le travail en groupe.

\subsubsection{MICHALON Loïc}

J'ai toujours aimé l'informatique. En effet, cela m'intéresse depuis que je suis tout petit notamment depuis la primaire. Cependant j'ai commencé à toucher au code en Seconde avec le Javascript mais n'ayant aucun projet à réaliser j'ai vite arrêté mais cela resta très formateur. J'ai donc surtout codé sur CASIO durant la Seconde et la Première, le temps me manquant en Terminale. Ainsi rentrer à l'EPITA m'a permis de me mettre pour de bon à l'informatique pour mon plus grand bonheur. \\
J'aime jouer aux jeux vidéos durant mon temps libre surtout aux jeux de stratégie ou aux jeux avec une grande rejouabilité tel que les rogue-like. En dehors des jeux vidéos j'aime bien m'instruire sur le monde extérieur en Histoire comme en géopolitique et j'affectionne tout particulièrement les sciences "dures". \\
Ce projet va me permettre d'avoir au moins fait un jeu dans ma vie. De plus, ce jeu fait parti d'un genre (rogue-like) que j'apprécie fortement et ce projet va donc me permettre de découvrir les coulisses de la programmation en équipe ce qui me semble très intéressant et formateur. Je regarde donc avec intérêt la suite de ce projet en espérant pouvoir le mener à bien lors du S2.

\subsubsection{MICHOT Maxence}

Passionné de programmation depuis la fin du collège, j'ai commencé avec des langages bas niveaux comme le C, le C++ ou encore le TI Basic. J'aime apprendre de nouveaux langages.

\subsubsection{TCHEKACHEV David}

...

\pagebreak

\section{Le projet}

\subsection{Origine et nature du projet}

D’où vient l’idée de ce projet, de quelle nature est il ? Jeu, Utilitaire, Traitement d’images, etc...

\subsubsection{Idée initiale}

...

\subsubsection{Nature du jeu}

...

\subsection{Objet de l'étude}

Quels sont les buts et intérêts de ce projet. Qu’est-ce que cela peut vous apporter en groupe ou individuellement ?

\subsection{État de l'art}

Quel est le premier logiciel/jeu de ce type ? Quels sont les principaux autres logiciels/jeux de ce type existants (vous en citerez au minimum trois) ? Quels sont leurs points forts ? Quelles sont leurs fonctionnalités propres ?

\subsection{Découpage du projet}

Qui, dans l’équipe qui réalisera ce projet fera quoi ? Partage des tâches,mais aussi découpage du projet en différentes parties si cela s’avère nécessaire. Par exemple pour un logiciel type Blender, il y aurait l’éditeur 2D, l’éditeur 3D, le gestionnaire de matériel,l’animateur, etc... \\
$\rightarrow$ $1^{er}$ Rogue Like \\
$\rightarrow$ Isaac, FTL, ETG, Hadès

\pagebreak

\subsubsection{Répartition des tâches}

\begin{table}[h!]
    \centering
    \caption*{Répartition des tâches par personne}
    \begin{tabular}{ |c|c|c|c|c| }
        \hline
        Tâches & Jean-Pierre & Loïc & Maxence & David \\
        \hline
        Progammation du jeu & - & - & - & - \\
        \hline
        Intelligence artificielle & - & - & - & - \\
        \hline
        Architecture des niveaux & - & - & - & - \\
        \hline
        Graphismes \& musiques & - & - & - & - \\
        \hline
        Réseau & - & - & - & - \\
        \hline
        Site Internet & - & - & $\times$ & $\otimes$ \\
        \hline
    \end{tabular}
    \caption*{
        \\ $\otimes$ : Responsable
        \\ $\times$ : Suppléant
    }
    \label{table:repartition}
\end{table}

\subsubsection{Logiciels et matériel utilisé}

...

\subsubsection{Le réseau}

Le jeu étant voué à être multijoueur, même si un mode solo sera disponible, il y aura deux entités séparées:

\begin{itemize}
    \item Le client:  Il sera chargé d'afficher le monde et interpréter les actions du joueur sur l'ordinateur. Il sert donc d'interface entre le serveur et le joueur.
    
    \item Le serveur: Il sera chargé de générer le monde, sauvegarder les parties des joueurs, interconnecter les joueurs lors des sessions multijoueurs et c'est lui qui va gérer la logique et le fonctionnement du jeu.
\end{itemize}

Lors des sessions solo, un serveur local est lancé sur l'ordinateur du joueur afin de garder la même base de code entre le mode solo et le mode multijoueur. \\
Ces deux entités communiquent entre elles en réseau, et s'échangent des données en continu pendant toute la partie.

\subsubsection{L'intelligence artificielle}

...

\end{document}
